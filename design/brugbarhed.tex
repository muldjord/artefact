\chapter{Brugbarhed}
Det er vigtigt at systemerne er nemme at benytte for brugerne, s� man
undg�r fejl, der opst�r pga. af forkert betjening. Eftersom
systemet vil blive benyttet adskillige gange dagligt af de samme
mennesker, er det dog samtidig vigtigt at denne let-tilg�ngelighed ikke
bliver en byrde for brugerne, s� de bliver tr�tte af at bruge systemet.\\
Det endelige brugbarheds princip er derfor at systemerne skal v�re
responsive og informative, men p� en ikkeforstyrrende m�de.\\
Det skal signaleres til brugeren, n�r data er sendt afsted og korrekt
modtaget p� en m�de som ikke generer arbejdsgangen og patienterne, og
samtidig uden at brugeren skal flytte sig, trykke p� noget eller p�
lignende vis aktivt g�re en indsats for at f� statusmeldinger.\\
Det kan f.eks. implementeres vha. af diskrete bippelyde, p� samme m�de
som det kendes fra supermarkedets kassesystem. Ved apparater, hvor
brugeren ser p� en sk�rm under produktionen af dataene, kan systemet
informere vha. lysdioder eller lign. monteret omkring sk�rmen, indenfor
synsvidde.\\
Udover denne umiddelbare respons skal der samtidig v�re mulighed for
at se status p� handlingen, hvis brugeren af een eller anden grund
skulle blive i tvivl om, hvorvidt den handling, der netop er foretaget,
rent faktisk var den forventede, alts� en form for visuel historik.\\
Et eksempel kan v�re, at en bruger har indtastet et CPR-nummer og
foretaget en m�ling p� et apparat, hvorefter brugeren bliver
i tvivl om, hvorvidt det rent faktisk var det rigtige CPR-nummer, der
blev sendt med. Her skal det kunne ses p� et display, at dataene er
korrekt modtaget p� serveren, samt hvilket CPR-nummer de er blevet knyttet til
i databasen, s� brugeren kan konstatere, om det er det rigtige.\\
En vigtig pointe med udarbejdelsen af de nye arbejdsprocedurer er, at
brugerne maksimalt g�r det samme som de g�r i \hyphenation{for-vej-en}forvejen - vi vil
s�ledes kun fjerne elementer fra arbejdsgangen eller erstatte
eksisterende elementer med nye, tilsvarende elementer. Det er
ekstremt vigtigt for oplevelsen af systemet, at brugerne opfatter det
som en forbedring i forhold til det, de havde i forvejen. For�get
funktionalitet som ikke er synlig for brugeren, s�som logning af
h�ndelser, m� s�ledes ikke blive introduceret p� bekostning af
arbejdsgangen.