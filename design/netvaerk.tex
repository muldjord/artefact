\chapter{Netv�rk}
Soekrisboksene er koblet op mod dataserveren over det normal
netv�rk, men p� sit eget net og s�ledes isoleret fra
produktionsnettet.

Dette kan g�res ved at s�tte to netkort i dataserveren, som s�ledes
kan modtage apparatdata p� det ene og svare p� foresp�rgsler p� det
andet.

Form�let med at separere nettene er, at vi ikke beh�ver kryptere eller
p� anden vis beskytte de data, som sendes fra apparaterne til
dataserveren. Disse data kan ellers betegnes som f�lsomme data, idet de ofte
vil indeholde f�lsomme personinformationer knyttet sammen med et
CPR-nummer.
Ydermere er det en fordel i forhold til at vedligeholde Soekris
boksene, at nettene er separerede, idet de ikke vil v�re under angreb
fra computervira og lignende.

Nettet skal selvf�lgelig v�re et kabelnet og ikke et tr�dl�st, idet hele
pointen med at det er afsk�rmet og dermed sikkert ellers ville
forsvinde.

Al kommunikation identificeres p� indhold og portnummer, og der skal
s�ledes allokeres faste portnumre til samtlige kommunikationstyper i
systemet.

Da der ikke er tale om store datam�ngder (modsat f.eks MIaV) kan
nettet bygges op af 100MBit enheder.